\documentclass[german]{article}

\usepackage{amsmath}
\usepackage{pst-circ}
\usepackage[utf8]{inputenc}
\usepackage{ulem}
\usepackage{color}

\newcommand{\ohm}{\Omega}

\begin{document}

%%29.11.2010

\section{Strom- und Spannungsmessung}

\subsection{Spannungsmesser}

$R_V = \infty, G_V = 0$

$R_V$: mit Verstärker $10M\ohm$ und mehr
			 ohne Verstärker $k\ohm/V$

\subsection{Strommesser}

ideal:
$R_A = 0$
$G_A = \infty$

real:
$R_A > 0$, z.B. $R_A = 150 m\ohm$

\subsection{Strommessung}

Ziel: Messen von $I_L$

richtiger Wert: $I_r = \frac{U_q}{R_i + R_L}$
angezeiger Wert: $I_a = \frac{U_q}{R_i + R_L + R_A}$

Systematischer Fehler:
Rel. Fehler: $F_R = \frac{I_a - I_r}{I_r} = \frac{\frac{U_q}{R_i + R_L + R_A} - \frac{U_q}{R_i + R_L}}{\frac{U_q}{R_i + R_L}} = \frac{-R_A}{R_i + R_L + R_A}$

1. Der relative Fehler verändert sich mit $R_A$

\subsection{Spannungsmessung}

Richtiger Wert: $U_r = \frac{I_q}{G_i + G_L}$
Angezeigter Wert: $U_a = \frac{I_q}{G_i + G_L + G_V}$
rel. Fehler: $F_r = \frac{-G_V}{G_i + G_L + G_V}$

\subsection{Gleichzeitiges Messen von Strom und Spannung}

\subsubsection{Stromrichtiges Messen}
%$(U_q + R_i) || V || (A + I_L + R_L)$

\begin{pspicture}(7,4)
	\pnode(1,1){A}
	\pnode(1,2.5){B}
	\pnode(1,4){C}
	\pnode(2.5,1){D}
	\pnode(2.5,4){E}
	\pnode(4,1){F}
	\pnode(4,4){G}
	\Ucc[tension,tensionlabel=$U_q$,tensionoffset=0.7,tensionlabeloffset=1](A)(B){}
	\resistor[labeloffset=0.5](B)(C){$R_i$}
	\circledipole[labeloffset=0](D)(E){\Large\textbf{V}}
	\circledipole[labeloffset=0](D)(F){\Large\textbf{A}}
	\resistor[labeloffset=0.5](G)(E){$R_L$}
	\wire[intensity,intensitylabel=$I_L$](F)(G)
	\wire(A)(D)
	\wire(C)(E)
\end{pspicture}

Angezeigter Wert: $U_a = U_{RL} + I_L R_A$

Anwendung: hochohmige Last $R_L$, kleiner Strom $I_L$

\subsubsection{Spannungsrichtiges Messen}
%(U_q + R_i + A + (V || G_L)

\begin{pspicture}(7,4)
	\pnode(1,1){A}
	\pnode(1,3){B}
	\pnode(3,3){C}
	\pnode(5,3){D}
	\pnode(5,1){E}
	\pnode(7,3){F}
	\pnode(7,1){G}
	\Ucc[tension,tensionlabel=$U_q$,tensionoffset=0.7,tensionlabeloffset=1](A)(B){}
	\resistor[labeloffset=0.5](B)(C){$R_i$}
	\circledipole[labeloffset=0](C)(D){\Large\textbf{A}}
	\circledipole[labeloffset=0](D)(E){\Large\textbf{V}}
	\resistor(F)(G){$G_L$}
	\wire(A)(E)
	\wire(D)(F)
	\wire(G)(E)
\end{pspicture}

Angezeigter Wert: $I_a = I_{L} + U_L G_V$

Anwendung: niederohmige Last $R_L$, kleine Spannung $U_L$

\section{Überlagerungsverfahren, Superpositionsverfahren}

\underline{Anwendungsbereich:} {\it Lineare Zeitinvariante} NW \\
\underline{Methode:} Man lässt jede Quelle einzeln wirken und überlagert die Teilwirkungen zur Gesamtwirkung.

(1) Alle im NW befindlichen Quellen werden bis auf eine als in dem Sinne nicht vorhanden angesehen, dass die Quellspannung $U_q$ und die Quellströme $I_q$ zu Null gesetzt werden.

(2) Ausgehen von der einzigen noch vorhandenen Quelle werden die benötigten Teilströme oder Teilspannungen berechnet. Für $n$ Quellen wird dieser Schritt $n$ mal ausgeführt.

(3) Teilströme und Teilspannungen werden {\it vorzeichenrichtig} überlagert oder addiert.
%bsp
\[U_1 = 9V, U_2 = 12V, R_1 = 2.4\ohm, R_2 = 1\ohm, R_3 = 2\ohm, R_4 = 3\ohm, R_5 = 5\ohm\]

(i) $U_2 = 0$

\[
	R_4* = R_4 || (R_3 + R_2 || R_5) = 1,4571\ohm
	I_1' = \frac{U_1}{R_1 + R_4*} = 2,3333 A
	U_4' = I_1'R_4* = 3,4V
	I_4' = \frac{U_4'}{R_4} = 1,13A
	I_3' = I_4' - I_1' = -1,2A
	I_2' = I_3' * \frac{G_2}{G_2 + G_5} = -1A
	I_5' = -I_3' * \frac{G_5}{G_2 + G_5} = 0,2A
\]

%% 03.12.2010

(ii) $U_1 = 0$

\begin{pspicture}(4,6)
	\pnode(0,5){A}
	\pnode(3,5){B}
	\pnode(0,4){C}
	\pnode(3,4){D}
	\pnode(0,1){E}
	\pnode(3,1){F}
	\resistor[labeloffset=0.5](A)(B){$R_1$}
	\resistor[labeloffset=0.5](C)(D){$R_4$}
	\resistor[labeloffset=0.5](C)(E){$R_3$}
	\resistor[labeloffset=0.5](D)(F){$R_2$}
	\resistor[labeloffset=0.5](E)(D){$R_5$}
	\Ucc[labeloffset=-0.85](E)(F){$\overrightarrow U_2$}
	\psline(A)(C)
	\psline(B)(D)
\end{pspicture}

\[
	R_5* = R_5 || ( R_3 + R_1 || R_4 ) = 2\ohm
	I_2'' = \frac{U_2}{R_2 + R_5*} = 4A
	U_5'' = I_2'' R_5 = 8V
	I_5'' = \frac{U_5''}{R_5} = 1,6A
\]

Aus $I_3'' + I_5'' = I_2''$ erhalten wir $I_3'' = 2,4A$

\[
	I_1'' = -I_3'' \frac{G_1}{G_1 + G_4} = -1,33A
	I_4'' = I_3'' \frac{G_4}{G_1 + G_4} = 1,07A
\]

(iii) Überlagern
\[
	I_1 = I_1' + I_1'' = 1,0 A
	I_2 = I_2' + I_2'' = 3,0 A
	I_3 = I_3' + I_3'' = 1,2 A
	I_4 = I_4' + I_4'' = 2,2 A
	I_5 = I_5' + I_5'' = 1,8 A
\]

%bsp
\begin{pspicture}(10,6)
	\pnode(1,5){A}
	\pnode(1,3){B}
	\pnode(1,1){C}
	\pnode(3,5){D}
	\pnode(3,3){E}
	\pnode(3,1){F}
	\pnode(5,5){G}
	\pnode(5,3){H}
	\pnode(5,1){I}
	\pnode(7,5){J}
	\pnode(7,3){K}
	\pnode(7,1){L}
	\pnode(9,5){M}
	\pnode(9,3){N}
	\pnode(9,1){O}
	\Ucc[tension,tensionlabel=$U_1$,tensionoffset=0.7,tensionlabeloffset=1](A)(B){}
	\resistor(B)(C){$R_1$}
	\Ucc[tension,tensionlabel=$U_2$,tensionoffset=0.7,tensionlabeloffset=1](D)(E){}
	\resistor(E)(F){$R_2$}
	\resistor(H)(I){$R_3$}
\end{pspicture}

\[
	U_1 = 40V, U_2 = 60V, I_6 = 1,2A, R_1 = R_3 = R_5 = 50\ohm, R_2 = R_4 = 40\ohm
\]
Berechne $I_3$

(i) $U_2 = 0$

\[
	R_2* = R_2 || (R_4 + R_5) = 27,6923\ohm
	I_1' = \frac{-U_1}{R_1 + R_3 || R_2*} = -0,5898 A
	I_3' = -I_1' \frac{G_3}{G_3 + G_2*} = 0,2102 A
\]

(ii) $U_1 = I_6 = 0$
\[
	R_1* = R_1 || (R_4 + R_5) = 32,1492\ohm
	I_2'' = \frac{-U_2}{R_2 + R_1* || R_3} = -1,0073 A
	I_3'' = -I_2'' \frac{G_3}{G_1* + G_3} = 0,3942 A
\]

(iii) $U_1 = U_2 = 0$

\[
	I_4''' = -I_6 \frac{G_4*}{G_4* + G_5} = - 0,5693 A
	I_3''' = -I_4 \frac{G_3}{G_1 + G_2 + G_3} = 0,1752 A
\]

$I_3 = I_3' + I_3'' + I_3''' = 0,7795 A$

\section{Topologische Methoden der NW-Analyse}

Ersetzt man in einem NW, das nur Zweipole enthält, jeden Zweipol symbolisch durch eine Linie, so erhält man einen Netzwerkgraphen:
\begin{itemize}
	\item Die Linien heißen {\it Zweige}.
	\item Die Punkte, in denen Zweige enden, heißen {\it Knoten}.
	\item Werden die Zweige mit Bezugsrichtungen gekennzeichnet, spricht man vom {\it gerichteten Graphen}.
	\item Die Zweige werden von $1$ bis $n$, die Knoten von $1$ bis $m$ nummeriert.
	\item {\it Maschen} sind Folgen von Zweigen mit der Eigenschaft, dass zwei aufeinander folgende Zweige in einem Knoten zusammentreffen und der erste und letzte Zweig einen Knoten gemeinsam haben. Bis auf den Start/End-Knoten darf beim Durchlaufen kein Knoten mehr als einmal angetroffen werden.
		Der Masche wird ein willkürlicher Umlaufsinn erteilt.
	\item Ein {\it vollständiger Baum} ist ein
		\begin{enumerate}
			\item zusammenhängender
			\item maschenfreier Untergraph, der
			\item alle Knoten der Graphen enthält.
		\end{enumerate}
	\item Die Zweige eines Baumes heißen {\it Äste} oder {\it Zweige}. Die nicht zum Baum gehörenden Zweige heißen {\it Sehnen} oder {\it Verbindungszweige}.
\end{itemize}

Im linearen NW kann man jeden Zweig-Zweipol als lineare Ersatzspannungsquelle oder als lineare Ersatzstromquelle darstellen. Zweige haben den Index $j$ mit $j=1,...,n$.
\begin{enumerate}
	\item
		\begin{pspicture}(3,3)
			\pnode(1,1){A}
			\pnode(3,1){B}
		\end{pspicture}
\end{enumerate}

\[U_j = U_{qj} + I_j R_j\]
$\downarrow$ Dualität
\[I_j = I_{qj} + U_j G_j\]

$U_j,I_j$: Zweigspannung, Zweigstrom
$U_{qj},I_{qj}$: Zweigquellspannung, Zweigquellstrom
$R_j, G_j$: Zweigwiderstand, Zweigleitwert

Ein Zweig mit Index $j$ wird entweder mit Zweigwiderstand $R_j$ und Zweigquellspannung $U_{qj}$ oder mit Zweigleitwert $G_j$ und Zweigquellstrom $I_{qj}$ charakterisiert.

\begin{tabular}{c||c|c|c|c|c}
	Zweigindex $j$ & $U_{qj}$ & $I_{qj}$ & $R_j$ & $G_j$ \\
	\hline
	$1$ & $U_{q1}$ & $\frac{U_{q1}}{R_1}$ & $R_1$ & $\frac{1}{R_1}$ \\
	$2$ & $0$ & $0$ & $R_1$ & $\frac{1}{R_1}$ \\
	$3$ & $U_{q3}$ & $\frac{U_{q3}}{R_3}$ & $R_3$ & $\frac{1}{R_3}$ \\
	$4$ & $0$ & $0$ & $R_4$ & $\frac{1}{R_4}$ \\
	$5$ & $0$ & $0$ & $R_5$ & $\frac{1}{R_5}$ \\
	$6$ & $0$ & $0$ & $R_6$ & $\frac{1}{R_6}$ \\
\end{tabular}

\section{Vollständiger Baum}

Von den $n$ Zweigen werden $m-1$ ausgewählt, um mit ihnen alle Knoten zu verbinden. Eine Bewegung von einem Knoten zu einem anderen hat genau einen Weg, entweder direkt oder über weitere Knoten. Die Teilmenge dieser Zweige bildet den vollständigen Baum.

\subsection{Methode des vollständigen Baums}

Eine Masche darf beliebig viele Äste aber nur eine \sout{Verbindungszweig} Sehne enthalten.

\[
	MGl_1: U_1 + U_2 + U_3 + U_4 = 0 \\
	MGl_2: -U_4 + U_5 + U_6 + U_7 = 0 \\
	MGl_3: U_1 + U_2 + U_3 + U_5 + U_7 + U_6 + U_3 = 0
\]

Wähle einen vollständigen Baum
%1,2,4,6,7
%=> dritte masche nicht verwendbar, da zwei sehnen enthalten

Ein vollständiger Graph hat für jedes Knotenpaar genau einen Zweig.
\[
	n = \sum_{k=1}^{m-1} k = \frac{m \cdot (m-1)}{2}
\]

$m = 4 \Rightarrow n=6$

Die Anzahl der vollständigen Bäume im vollständigen Graphen beträgt: $B_{ANZ} = m^{m-2}$

$m = 4 \Rightarrow B_{ANZ} = 16$

\vspace{1cm}

%alle 16 vollständigen Bäume des Graphen mit 4 Knoten (siehe kua^^)
\begin{pspicture}(1,1)
	\pnode(0,0){A}
	\pnode(1,0){B}
	\pnode(0,1){C}
	\pnode(1,1){D}
	\wire(A)(B)
	\wire(A)(C)
	\wire(A)(D)
\end{pspicture}
\hspace{2cm}
\begin{pspicture}(1,1)
	\pnode(0,0){A}
	\pnode(1,0){B}
	\pnode(0,1){C}
	\pnode(1,1){D}
	\wire(A)(B)
	\wire(B)(C)
	\wire(B)(D)
\end{pspicture}
\hspace{2cm}
\begin{pspicture}(1,1)
	\pnode(0,0){A}
	\pnode(1,0){B}
	\pnode(0,1){C}
	\pnode(1,1){D}
	\wire(A)(C)
	\wire(C)(B)
	\wire(C)(D)
\end{pspicture}
\hspace{2cm}
\begin{pspicture}(1,1)
	\pnode(0,0){A}
	\pnode(1,0){B}
	\pnode(0,1){C}
	\pnode(1,1){D}
	\wire(D)(B)
	\wire(D)(C)
	\wire(A)(D)
\end{pspicture}

\vspace{2cm}

\begin{pspicture}(1,1)
	\pnode(0,0){A}
	\pnode(1,0){B}
	\pnode(0,1){C}
	\pnode(1,1){D}
	\wire(A)(B)
	\wire(B)(D)
	\wire(A)(C)
\end{pspicture}
\hspace{2cm}
\begin{pspicture}(1,1)
	\pnode(0,0){A}
	\pnode(1,0){B}
	\pnode(0,1){C}
	\pnode(1,1){D}
	\wire(A)(B)
	\wire(B)(D)
	\wire(D)(C)
\end{pspicture}
\hspace{2cm}
\begin{pspicture}(1,1)
	\pnode(0,0){A}
	\pnode(1,0){B}
	\pnode(0,1){C}
	\pnode(1,1){D}
	\wire(A)(B)
	\wire(C)(D)
	\wire(C)(A)
\end{pspicture}
\hspace{2cm}
\begin{pspicture}(1,1)
	\pnode(0,0){A}
	\pnode(1,0){B}
	\pnode(0,1){C}
	\pnode(1,1){D}
	\wire(A)(C)
	\wire(B)(D)
	\wire(C)(D)
\end{pspicture}

\vspace{2cm}

\begin{pspicture}(1,1)
	\pnode(0,0){A}
	\pnode(1,0){B}
	\pnode(0,1){C}
	\pnode(1,1){D}
	\wire(A)(B)
	\wire(C)(D)
	\wire(B)(C)
\end{pspicture}
\hspace{2cm}
\begin{pspicture}(1,1)
	\pnode(0,0){A}
	\pnode(1,0){B}
	\pnode(0,1){C}
	\pnode(1,1){D}
	\wire(A)(C)
	\wire(B)(D)
	\wire(A)(D)
\end{pspicture}
\hspace{2cm}
\begin{pspicture}(1,1)
	\pnode(0,0){A}
	\pnode(1,0){B}
	\pnode(0,1){C}
	\pnode(1,1){D}
	\wire(A)(B)
	\wire(C)(D)
	\wire(A)(D)
\end{pspicture}
\hspace{2cm}
\begin{pspicture}(1,1)
	\pnode(0,0){A}
	\pnode(1,0){B}
	\pnode(0,1){C}
	\pnode(1,1){D}
	\wire(A)(C)
	\wire(B)(D)
	\wire(B)(C)
\end{pspicture}

\vspace{2cm}

\begin{pspicture}(1,1)
	\pnode(0,0){A}
	\pnode(1,0){B}
	\pnode(0,1){C}
	\pnode(1,1){D}
	\wire(A)(B)
	\wire(A)(D)
	\wire(B)(C)
\end{pspicture}
\hspace{2cm}
\begin{pspicture}(1,1)
	\pnode(0,0){A}
	\pnode(1,0){B}
	\pnode(0,1){C}
	\pnode(1,1){D}
	\wire(B)(D)
	\wire(B)(C)
	\wire(A)(D)
\end{pspicture}
\hspace{2cm}
\begin{pspicture}(1,1)
	\pnode(0,0){A}
	\pnode(1,0){B}
	\pnode(0,1){C}
	\pnode(1,1){D}
	\wire(C)(D)
	\wire(C)(B)
	\wire(A)(D)
\end{pspicture}
\hspace{2cm}
\begin{pspicture}(1,1)
	\pnode(0,0){A}
	\pnode(1,0){B}
	\pnode(0,1){C}
	\pnode(1,1){D}
	\wire(A)(C)
	\wire(A)(D)
	\wire(B)(C)
\end{pspicture}

%% 06.12.2010

\section{NW-Analyse mit dem Verfahren des vollständigen Baums}

\begin{enumerate}
	\item NW-Zahlwerte \\
		$m$: Anzahl der Knoten, $i = 1, ..., m$ \\
		$n$: Anzahl der Zweige, $j = 1, ..., n$
	\item Baumzahlwerte \\
		$k = m-1$: Anzahl der Äste im vollst. Baum, Anz. der Knotengleichungen \\
		$l = n - m + 1$: Anzahl der Sehnen, Anzahl der Maschengleichungen \\
		$l + k = n$
	\item Zur Bestimmung von $n$ Zweiggrößen, Zweigspannungen und Zweigströme, mit unterschiedlichen Indices, benötigt man $n$ unabhängige Gleichungen:
		\begin{enumerate}
			\item $k = m-1$ Knotengleichungen, in dem man einen Knoten ``wegläßt'', zum Bezugsknoten erklärt
			\item Die fehlenden $l = n - m + 1$ Machengleichungen werden mit dem Verfahren des vollständigen Baums beschafft.
		\end{enumerate}
\end{enumerate}

\begin{pspicture}(8,6)
	\pnode(1,1){A}
	\pnode(1,3.5){B}
	\pnode(1,6){C}
	\pnode(6,6){D}
	\pnode(3.5,3.5){E}
	\pnode(4.5,3.5){E2}
	\pnode(8.5,3.5){F}
	\pnode(7.5,3.5){F2}
	\pnode(6,3.5){G}
	\pnode(6,1){H}
	\Ucc(A)(B){$U_{q1}$}
	\resistor[labeloffset=0.6](B)(C){$R_1$}
	\wire(C)(D)
	\resistor[labeloffset=-0.6](D)(E){$R_6$}
	\resistor[labeloffset=0.6](D)(F){$R_5$}
	\resistor[labeloffset=0.6](E2)(G){$R_3$}
	\wire(E)(E2)
	\Ucc(G)(F2){$U_{q3}$}
	\wire(F)(F2)
	\resistor[labeloffset=-0.6](E)(H){$R_2$}
	\resistor[labeloffset=0.6](F)(H){$R_4$}
	\wire(H)(A)
\end{pspicture}

\paragraph{Schritt 1}
	Erstellung des gerichteten Graphen, Nummeriertung der Knoten und Zweige

	\begin{pspicture}(8,7)
		\pnode(1,1){A}
		\pnode(1,6){C}
		\pnode(6,6){D}
		\pnode(3.5,3.5){E}
		\pnode(4.5,3.5){E2}
		\pnode(8.5,3.5){F}
		\pnode(7.5,3.5){F2}
		\pnode(6,3.5){G}
		\pnode(6,1){H}
		\uput[180](E){\pscirclebox{1}}
		\uput[90](D){\pscirclebox{2}}
		\uput[0](F){\pscirclebox{3}}
		\uput[270](H){\pscirclebox{4}}
		\wire[intensity,intensitywidth=0.1,intensitylabel=$1$](A)(C)
		\wire(C)(D)
		\wire[intensity,intensitywidth=0.1,intensitylabel=$6$,intensitylabeloffset=-0.5](D)(E)
		\wire[intensity,intensitywidth=0.1,intensitylabel=$5$](D)(F)
		\wire(E2)(G)
		\wire[intensity,intensitywidth=0.1,intensitylabel=$3$](E)(F)
		\wire[intensity,intensitywidth=0.1,intensitylabel=$2$,intensitylabeloffset=-0.5](E)(H)
		\wire[intensity,intensitywidth=0.1,intensitylabel=$4$](F)(H)
		\wire(H)(A)
	\end{pspicture}

\paragraph{Schritt 2}
	Festlegung des vollständigen Baums\\
	\textcolor{blue}{Sehnen in blau}\\
	\textcolor{red}{Äste in rot}

	\begin{pspicture}(8,7)
		\pnode(1,1){A}
		\pnode(1,6){C}
		\pnode(6,6){D}
		\pnode(3.5,3.5){E}
		\pnode(4.5,3.5){E2}
		\pnode(8.5,3.5){F}
		\pnode(7.5,3.5){F2}
		\pnode(6,3.5){G}
		\pnode(6,1){H}
		\uput[180](E){\pscirclebox{1}}
		\uput[90](D){\pscirclebox{2}}
		\uput[0](F){\pscirclebox{3}}
		\uput[270](H){\pscirclebox{4}}
		\wire[intensity,intensitywidth=0.1,intensitylabel=$1$,intensitycolor=red,linecolor=red](A)(C)
		\wire[linecolor=red](C)(D)
		\wire[intensity,intensitywidth=0.1,intensitylabel=$6$,intensitylabeloffset=-0.5,intensitycolor=blue,linecolor=blue](D)(E)
		\wire[intensity,intensitywidth=0.1,intensitylabel=$5$,intensitycolor=blue,linecolor=blue](D)(F)
		\wire[intensity,intensitywidth=0.1,intensitylabel=$3$,intensitycolor=red,linecolor=red](E)(F)
		\wire[intensity,intensitywidth=0.1,intensitylabel=$2$,intensitylabeloffset=-0.5,intensitycolor=red,linecolor=red](E)(H)
		\wire[intensity,intensitywidth=0.1,intensitylabel=$4$,intensitycolor=blue,linecolor=blue](F)(H)
		\wire[intensitycolor=red,linecolor=red](H)(A)
	\end{pspicture}

\paragraph{Schritt 3}
	Aufstellung der Inzidenzmatrix, Knotenmatrix \\
	Die vollständige ($m x n$)-Matrix $H_v$ verschreibt die NW-Struktur. Sie hat die Elemente $h_{ij},
	h_{ij} \in \{-1,0,1\}$ mit $h_{ij} = \left\{ \begin{array}{c}1\\ -1\\ 0 \end{array} \right\}$, wenn der Zweigstrom mit Index $j$
	\[
		\left\{
			\begin{array}{c}
				\text{vom Knoten } i \text{ wegführt}\\
				\text{zum Knoten } i \text{ hinführt}\\
				\text{mit dem Knoten } i \text{ nicht inzident ist}
			\end{array}
		\right\}
	\]

	\[
		\begin{array}{cccccc}
			0 & 1 & 1 & 0 & 0 & -1 \\
			-1 & 0 & 0 & 0 & 1 & 1 \\
			0 & 0 & -1 & 1 & -1 & 0 \\
			1 & -1 & 0 & -1 & 0 & 0
		\end{array}
	\]
	Lezte Kontrollmöglichkeit: Die Spaltensummen müssen Null ergeben

	In $H_v$ sind nur $m-1$ Zeilen unabhängig von einander. Wir gewinnen die reduzierte Matrix $H$, indem wir einen Knoten zum Bezugsknoten wählen. Wahl: Knoten $4$ ist Bezugsknoten.

	$H$ wird gewonnen, in dem in $H_v$ die dem Bezugsknoten entsprechende Zeile gestrichen wird.

	\[
		\left(
		\begin{array}{cccccc}
			0 & 1 & 1 & 0 & 0 & -1 \\
			-1 & 0 & 0 & 0 & 1 & 1 \\
			0 & 0 & -1 & 1 & -1 & 0 \\
		\end{array}
		\right)
		\left(
		\begin{array}{c}
			I_1 \\
			I_2 \\
			I_3 \\
			I_4 \\
			I_5 \\
			I_6
		\end{array}
		\right)
		=
		\left(
		\begin{array}{c}
			0 \\
			0 \\
			0
		\end{array}
		\right)
	\]
	\[
		HI = 0
	\]

\paragraph{Schritt 4}
	Beschaffung der $l - n-m+1$ noch benötigten Machengleichungen

	\underline{Satz:} Zu jeder Sehne werden Äste so hinzugenommen, daß sie zusammen eine Masche bilden. Die Umlaufrichrung der Masche wird durch die Richtung der erzeugenden Sehne bestimmt. Mit diesen Maschen wird die $((n-m+1) x (n))$-Machenmatrix $M$ aufgestellt, mit den Elementen
	$m_{ij} = \left\{ \begin{array}{c} 1\\ -1\\ 0 \end{array} \right\}$ wenn der Zweig $j$
	\[
		\left\{
			\begin{array}{c}
				\text{zur Masche } i \text{ gehört und in seiner Richtung durchlaufen wird.}\\
				\text{zur Masche } i \text{ gehört und entgegen seiner Richtung durchlaufen wird.}\\
				\text{nicht zur Masche } i \text{ gehört.}
			\end{array}
		\right\}
	\]

	Masche 1, bzgl. Sehne mit Index 4

	%4, 2, 3

	Masche 2, bzgl. Sehne mit Index 5

	%5, 3, 2, 1

	Masche 3, bzgl. Sehne mit Index 6

	%6, 1, 2
	\[
		\begin{pmatrix}
			0 & -1 & 1 & 1 & 0 & 0 \\
			1 & 1 & -1 & 0 & 1 & 0 \\
			1 & 1 & 0 & 0 & 0 & 1 \\
		\end{pmatrix}
	\]

	Die Maschengleichungen lauten dann: $MU = 0$

	\[
		\begin{pmatrix}
			0 & -1 & 1 & 1 & 0 & 0 \\
			1 & 1 & -1 & 0 & 1 & 0 \\
			1 & 1 & 0 & 0 & 0 & 1 \\
		\end{pmatrix}
		\begin{pmatrix}
			U_1 \\
			U_2 \\
			U_3 \\
			U_4 \\
			U_5 \\
			U_6
		\end{pmatrix}
		=
		\begin{pmatrix}
			0 \\
			0 \\
			0
		\end{pmatrix}
	\]

%TODO: hier fehlt noch was

%%10.12.2010

	Gleichungssystem für $n$ Zweigströme.
	\[
		\begin{pmatrix}
			H \\
			---- \\
			M \cdot diag(R)
		\end{pmatrix}
		\cdot I =
		\begin{pmatrix}
			0 \\
			--- \\
			-M U_q
		\end{pmatrix}
	\]
	Gleichungssystem für $n$ Zweigspannungen
	\[
		\begin{pmatrix}
			H \cdot diag(G) \\
			---- \\
			M
		\end{pmatrix}
		\cdot U =
		\begin{pmatrix}
			-H I_q \\
			--- \\
			0
		\end{pmatrix}
	\]

	Brückenschaltungsbeispiel
	\[
		%bla
	\]

	\paragraph{Zahlenbeispiel}
	
	%Schaltplan -> Daniel

	\[
		\begin{array}{cc}
			U_{q2} = 300V & R_2 = 0.25\ohm \\
			U_{q3} = 270V & R_3 = 0.12\ohm \\
			R_1: \text{ Verbraucher}
		\end{array}
	\]

	Wie verteilen sich die Ströme $I_1$, $I_2$ und $I_3$, wenn der Verbraucherwiderstand $R_1$ im Bereich $0 < R_1 < 10\ohm$ gewählt wird?

	$n = 3, m = 2 \Rightarrow 1$ KGl, $2$ MGl

	Gerichteter Graph
	%\begin{pspicture}
	%	\pnode(3)(1){A}
	%	\pnode(3)(3){B}
	%	\wire[intensity](B)(A){2}
	%	\wire[intensity](B)(A){1}
	%	\wire[intensity](B)(A){3}
	%\end{pspicture}

	Vollständiger Baum
	%1 als Sehne

	Inzidenzmatrix $(1x3)$-Vektor
	\[
		H =
		\begin{pmatrix}
			1 & -1 & -1
		\end{pmatrix}
	\]
	Bezugsknoten $2$

	Maschenmatrix
	%masche 1 (2,1) und masche 2 (1,3) als graphen

	\[
		M = 
		\begin{pmatrix}
			1 & 1 & 0 \\
			1 & 0 & 1
		\end{pmatrix}
	\]

	Gleichungssysteme
	\begin{eqnarray*}
		H \cdot I & = & 0 \\
		M \cdot U & = & 0
	\end{eqnarray*}

	Gleichuungssysteme für die Zweigströme in $I$
	\[
		M^* = M \cdot diag(R) =
			\begin{pmatrix}
				1 & 1 & 0 \\
				1 & 0 & 1
			\end{pmatrix}
			\begin{pmatrix}
				R_1 & 0 & 0 \\
				0 & R_2 & 0 \\
				0 & 0 & R_3
			\end{pmatrix}
			=
			\begin{pmatrix}
				R_1 & R_2 & 0 \\
				R_1 & 0 & R_3
			\end{pmatrix}
	\]
	\[
		M \cdot U_q =
			\begin{pmatrix}
				1 & 1 & 0 \\
				1 & 0 & 1
			\end{pmatrix}
			\begin{pmatrix}
				0 \\
				-U_{q2} \\
				-U_{q3}
			\end{pmatrix}
			=
			\begin{pmatrix}
				- U_{q2} \\
				- U_{q3}
			\end{pmatrix}
	\]
	\[
		\text{Also: }
		\begin{pmatrix}
			1 & -1 & -1 \\
			R_1 & R_2 & 0 \\
			R_1 & 0 & R_3
		\end{pmatrix}
		\begin{pmatrix}
			I_1 \\
			I_2 \\
			I_3
		\end{pmatrix}
		=
		\begin{pmatrix}
			0 \\
			U_{q2} \\
			U_{q3}
		\end{pmatrix}
	\]

	Gleichungssystem für die Zweigspannungen in $U$
	\[
		H^* = H \cdot diag(G) =
			\begin{pmatrix}
				1 & -1 & -1 \\
			\end{pmatrix}
			\begin{pmatrix}
				G_1 & 0 & 0 \\
				0 & G_2 & 0 \\
				0 & 0 & G_3
			\end{pmatrix}
			=
			\begin{pmatrix}
				G_1 & -G_2 & -G_3
			\end{pmatrix}
	\]
	\[
		H \cdot I_q =
			\begin{pmatrix}
				1 & -1 & -1
			\end{pmatrix}
			\begin{pmatrix}
				0 \\
				I_{q2} \\
				I_{q3}
			\end{pmatrix}
			=
			\begin{pmatrix}
				- I_{q2} - I_{q3}
			\end{pmatrix}
			=
			\begin{pmatrix}
				\frac{U_{q2}}{R_2} + \frac{U_{q3}}{R_3}
			\end{pmatrix}
	\]
	\[
		\text{Also: }
		\begin{pmatrix}
			G_1 & -G_2 & -G_3 \\
			1 & 1 & 0 \\
			1 & 0 & 1
		\end{pmatrix}
		\begin{pmatrix}
			U_1 \\
			U_2 \\
			U_3
		\end{pmatrix}
		=
		\begin{pmatrix}
			- \frac{U_{q2}}{R_2} - \frac{U_{q3}}{R_3} \\
			0 \\
			0
		\end{pmatrix}
	\]

\paragraph{Komprimierte Verfahren}
	\begin{itemize}
		\item Berechnung von Sehnenströmen {\it (Machenstromverfahren)}
		\item Berechnung der Knotenpotenziale {\it (Knotenpotenzialverfahren)}
		\item Berechnung der Astspannungen {\it (Astspannungsverfahren)}
	\end{itemize}

\paragraph{Maschenstromverfahren}
	\begin{enumerate}
		\item Das NW wird allein mit Widerständen und Spannungsquellen dargestellt.
		\item Der gerichtete Graph wird erstellt. Ein vollst. Baum wird so gewählt, daß die interessierenden Zweigströme Sehnenströme sind.
		\item Maschen aufstellen. Die MGl dürfen nur Shenenströme einhalten.
		\item Falls Astströme benötigt werden, berechnen wir sie aus den Sehnenströmen.
	\end{enumerate}

%Schaltung siehe Daniel

Berechne die Zweigströme $I_1$, $I_2$ und $I_3$

Knotengleichungen

$I_4 = I_1 - I_3$
$I_5 = I_2 + I_3$

Masche 1

$U_1 + U_4 = 0$
$-U_{q1} + I_1 R_1 + I_4 R_4 = 0$
...
$I_1(R_1 + R_4) - I_3 R_3 - U_{q1} = 0$

Masche 2

$U_3 - U_4 + U_5 = 0$
$I_3 R_3 - I_4 R_4 + I_5 R_5 = 0$
$-I_1 R_4 + I_2 R_5 + I_3(R_3 + R_4 + R_5) = 0$

Masche 3

$U_2 + U_5 = 0$
$I_2 R_2 + I_5 R_5 = 0$
$I_2 ( R_2 + R_5 ) + I_3 R_5 - U_{q2} = 0$

\[
	\begin{pmatrix}
		R_1 + R_4 & 0 & -R_4 \\
		-R_4 & R_5 & R_3 + R_4 + R_5 \\
		0 & R_2 + R_5 & R_5
	\end{pmatrix}
	\begin{pmatrix}
		I_1 \\
		I_2 \\
		I_3
	\end{pmatrix}
	=
	\begin{pmatrix}
		U_{q1} \\
		0 \\
		U_{q2}
	\end{pmatrix}
\]

\[
	\begin{pmatrix}
		R_1 + R_4 & 0 & -R_4 \\
		0 & R_2 + R_5 & R_5 \\
		-R_4 & R_5 & R_3 + R_4 + R_5
	\end{pmatrix}
	\begin{pmatrix}
		I_1 \\
		I_2 \\
		I_3
	\end{pmatrix}
	=
	\begin{pmatrix}
		U_{q1} \\
		U_{q2} \\
		0
	\end{pmatrix}
\]

\[
	R^* \cdot I_S = U_q^*
\]
($I_S$ Vektor mit den Sehnenströmen)

\paragraph{Beobachtungen}
\begin{enumerate}
	\item Hauptdiagonale $r_{ij}^*$, $i = j$, $i = 1, 2, \ldots, n-m+1$ Summen der in den jeweiligen Maschen liegenden Widerstände
	\item Nebendiagonalen $r_{ij}^*$, $i \neq j$, $i = 1, 2, \ldots, n-m+1$, $j = 1, 2, \ldots, n - m + 1$, $r_{ij}^* = r_{ji}^*$ Widerstandssummen für die Widerstände, die die Maschen $i$ und $j$ gemeinsam haben. Positiv wird bei gleichsinnigem Durchlauf gerechnet.
	\item Im Vektor $U_q^*$ werden die Summen der Maschenquellspannungen eingetragen. Positiv wird bei gegensinnigem Durchlauf gerechnet.
\end{enumerate}

%Schaltplan siehe Daniel

\[
	R^* = 
	\begin{pmatrix}
		r_{11}^* & r_{12}^* & ... \\
		r_{21}^* & r_{22}^* & ... \\
		... & ... & ...
	\end{pmatrix}
\]
\begin{eqnarray*}
	r_{11}^* & = & R_2 + R_3 + R_4 + R_5 + R_6 \\
	r_{12}^* & = & R_2 + R_3 + R_4 = R_{21}^* \\
	r_{22}^* & = & R_1 + R_2 + R_3 + R_4 + R_7
\end{eqnarray*}

%%13.12.2010

%beispiel fehlt

Wir wollen im weiteren die NW-Berechnung verallgemeinern und Berechnungsverfahren mit Dualitätsbeziehungen gewinnen. So gewinnen wir die Astspannungsanalyse aus der Sehnenstromalanyse (Maschenstromanalyse). Für diese Betrachtungen benötigen wir eine dritte Matrix, die mit der Knotenmatrix verwandte Schnittmatrix S (das ist eine Superknotenmatrix). Wir sorgen dafür, dass alle NW-Vektoren und NW-Matrizen in der Reihenfolge
{\red Astgrößen} - {\blue Sehnengrößen}
geordnet werden.

\[
	S = \begin{pmatrix}S_A & | & S_S\end{pmatrix}
\]
\[
	I = \begin{pmatrix}
		I_A \\
		--- \\
		I_S
	\end{pmatrix}
\]
\[
	U = \begin{pmatrix}
		U_A \\
		--- \\
		U_S
	\end{pmatrix}
\]
\[
	I_q = \begin{pmatrix}
		I_{qA} \\
		--- \\
		I_{qS}
	\end{pmatrix}
\]
\[
	U_q = \begin{pmatrix}
		U_{qA} \\
		--- \\
		U_{qS}
	\end{pmatrix}
\]
\[
	H = \begin{pmatrix}
		H_A & | & H_S
	\end{pmatrix}
\]
\[
	M = \begin{pmatrix}
		M_A & | & M_S
	\end{pmatrix}
\]

%graph 1 - 2 - 3 | 4 -- 1:31, 2:12, 3:32, 4:14, 5:42, 6:43

vollst. Baum mit den Zweigen 1, 3 und 5

%graph hier

Wir sortieren die Zweige in der Reihenfolge
{\red 1, 3, 5, }{\blue 2, 4, 6}
und erden den Knoten 4.

\paragraph{Schnittmatrix}
Ein Schnitt teilt ein NW in zwei Teil-NW. Wir suchen die Tundamentalschnitte, die {\underline einen} Ast und im weiteren nur Sehnen schneiden. Das ist dual zur Masche, die {\underline eine} Sehne und im weitern nur Äste enthält. Die Anzahl der (Fundamental-) Schnitte entspricht der Anzahl der Äste $m-1$. Die Orientierung des Schnitts entspricht der Orientierung des Bezugsasts.

%S_1: schneidet 1,2 und 4
%S_2: schneidet 2,3,4 und 6
%S_3: schneidet 4,5 und 6

Schnitt $S_1$
%NW_1: Knoten 1, NW_2 Knoten 2,3,4 (Superknoten), Verbunden über 1,2 und 4
Kirchhoff: $I_1 = I_2 + I_4$

\[
	S =
	\begin{pmatrix}
		{\red 1} & {\red 0} & {\red 0} & | & {\blue -1} & {\blue -1} & {\blue 0} \\
		{\red 0} & {\red 1} & {\red 0} & | & {\blue 1} & {\blue 1} & {\blue -1} \\
		{\red 0} & {\red 0} & {\red 1} & | & {\blue 0} & {\blue -1} & {\blue 1}
	\end{pmatrix} =
	\begin{pmatrix}
		E & | & S_S
	\end{pmatrix}
\]

E: Einheistmatrix in der passenden Dimension
\[
	\begin{pmatrix}
		1 & 0 & 0 & \cdots & 0 \\
		0 & 1 & 0 & \cdots & 0 \\
		\vdots & 0 & \ddots & 0 & \vdots \\
		0 & \cdots & \cdots & 0 & 1
	\end{pmatrix}
\]

\paragraph{Maschenmatrix $M$}

\[
	M = \begin{pmatrix} M_A & | & M_S \end{pmatrix}
	\begin{pmatrix}
		{\red 1} & {\red -1} & {\red 0} & | & {\blue 1} & {\blue 0} & {\blue 0} \\
		{\red 1} & {\red -1} & {\red 1} & | & {\blue 0} & {\blue 1} & {\blue 0} \\
		{\red 0} & {\red 1} & {\red -1} & | & {\blue 0} & {\blue 0} & {\blue 1}
	\end{pmatrix} = \begin{pmatrix} M_A & | & E \end{pmatrix}
\]

\[
	H = \begin{pmatrix} H_A & | & H_S \end{pmatrix} =
	\begin{pmatrix}
		{\red -1} & {\red 0} & {\red 0} & | & {\blue 1} & {\blue 1} & {\blue 0} \\
		0 & -1 & -1 & | & -1 & 0 & 0 \\
		1 & 1 & 0 & | & 0 & 0 & -1 \\
	\end{pmatrix}
\]

Kirchhoffsche Gesetze

HI = 0 Die Summe der Ströme im Knoten ist Null
SI = 0 Die Summe der Ströme im Superknoten ist Null
MU = 0 Die Maschenumlaufspannung ist Null

\paragraph{Zusammenhang zwischen Schnitt- und Inzidenzmatrix}
\[
	\left. \begin{array}{l}
	SI = 0 \Rightarrow \begin{pmatrix} S_A & | & S_S \end{pmatrix} \begin{pmatrix} I_A \\ --- \\ I_S \end{pmatrix} = S_A I_A + S_S I_S = E I_A + S_S I_S = I_A + S_S I_S = 0 \Rightarrow I_A = - S_S I_S
	HI = 0 \Rightarrow \begin{pmatrix} H_A & | & H_S \end{pmatrix} \begin{pmatrix} I_A \\ --- \\ I_S \end{pmatrix} = H_A I_A + H_S I_S = 0 \Rightarrow I_A = -H_A^{-1} H_S I_S
	\end{array} \right\} \Rightarrow S_S = H_A^{-1} H_S
\]
Also $S = \begin{pmatrix} S_A & | & S_S \end{pmatrix} = \begin{pmatrix} E & | & H_A^{-1} H_S \end{pmatrix}$

\paragraph{Zusammenhang zwischen Maschen- und Inzidenzmatrix}
\[
	MU = 0 + \begin{pmatrix} M_A & | & \underbrace{M_S}_{=E} \end{pmatrix} \begin{pmatrix} U_A \\ --- \\ U_S \end{pmatrix} = M_A U_A + U_S = 0 \Rightarrow U_S = - M_A U_A
\]
\[
	H M^T = \begin{pmatrix} H_A & | & H_S \end{pmatrix} \begin{pmatrix} M_A^T \\ --- \\ M_S^T \end{pmatrix} = \begin{pmatrix} H_A & | & H_S \end{pmatrix} \begin{pmatrix} M_A^T \\ --- \\ E \end{pmatrix} = H_A M_A^T + H_S E = 0 \Rightarrow M_A^T = -H_A^{-1} H_S = -S_S
\]

Zusammenfassung: Wir müssen nur die Inzidenzmatrix berechnen und gewinnen
\[
	M = \begin{pmatrix} ( -H_A^{-1} H_S )^T & | & E \end{pmatrix} = \begin{pmatrix} - S_S^T & | & E \end{pmatrix}
\]
\[
	S = \begin{pmatrix} E & | & -H_A^{-1} H_S \end{pmatrix} = \begin{pmatrix} E & | & - M_A^T \end{pmatrix}
\]

Mit diesen Erkenntnissen definieren wir das Maschenstromverfahren

Ausgangspunkt: $M \cdot diag(R) \cdot I = -M \cdot U_q$ Aus $I = M^T I_S$, folgt aus $S M^T = 0$
\[
	\underbrace{M diag(R) M^T}_{R^*} I_S = \underbrace{-M U_q}_{U_q^*}
\]
\[
	I_A = M_A^T I_S = ( -H_A^{-1} H_S ) I_S = - S_S I_S
\]

Astspannungsanalyse
\[
	S diag(G) S^T U_A = -S I_q
\]
\[
	U_S = S_S^T U_A = ( -H_A^{-1} H_S ) U_A = - M_A U_A
\]

%%17.12.2010
\paragraph{Schnittmatrix für die Brückenschaltung}
%brückenschaltung mit 1-4, 4-2 und 2-3 als vollständiger Baum

Fundamentalschnitte
%S_1: knoten 1
%S_2: knoten 2 und 3
%S_3: knoten 3

\[
	S =
	\begin{pmatrix}
		{\red 1} & 0 & 0 & | & {\blue 0} & -1 & -1 \\
		0 & 1 & 0 & | & 1 & -1 & -1 \\
		0 & 0 & 1 & | & -1 & 1 & 0
	\end{pmatrix}
\]

Dualitätsbeziehungen
\begin{tabular}{c|c|c}
	Maschenstromverfahren & & Astspannungsverfahren \\
	 & & Knotenspannungsverfahren (nicht so richtig dual, da Knotenpotenziale nicht notwendigerweise Zweigspannungen) \\
	 Anzahl der Sehnen (n-m+1) & Anzahl der Gleichungen & Anzahl der Äste (m-1) \\
	 Masche &  & Superknoten / Knoten \\
	 Strom & & Spannung \\
	 Sehne & & Ast \\
	 Alle Zweige enthalten nur Spannungsquellen und Widerstände & & Alle Zweige enthalten nur Stromquellen und Leitwerte \\
	 $M \cdot diag(R) M^T I_S = - M U_q$ & & $S \cdot diag(G) S^T U_A = -S I_q$ Astspannungsverfahren \\
	 $I_A = M_A^T I_S$ & & $U_S = S_S^T U_A$ \\
	  & & $H \cdot diag(G) H^T U_K = -H I_q$ Knotenspannungsverfahren \\
		& & $U = H^T U_K$ $U_K$ Vektor der Knotenpotenziale \\
	Summe der Widerstände in einer Masche & & Summe der Leitwerte am (Super-) Knoten \\
	Widerstände zwischen den Maschen & & Leitwerte zwischen den (Super-) Knoten \\
	Summe der Quellspannungen in einer Masche & & Summe der Quellströme an einem (Super-) Knoten
\end{tabular}

Knotenpotenziale - Astspannungen
%Brückenschaltung 2 | 1 - 3 | 4
%U_1: 4-2, U_2: 4-1, U_3: 1-3, U_4: 4-3, U_5: 2-3, U_6: 2-1
%U_K_1: 1-4, U_K_2: 2-4, U_K_3: 3-4

{\color{orange} $U_1$, $U_2$, ..., $U_6$: Astspannungen} \\
{\color{red} $U_1$, $U_2$, $U_3$: Astspannungen} \\
{\color{blue} $U_4$, $U_5$, $U_6$: Sehnenspannungen} \\
Bezugsknoten: 4 \\
{\color{yellow} $U_{K_1}$, $U_{K_2}$, $U_{K_3}$: Knotenpotentiale}

Knotenpotential
\begin{enumerate}
	\item Erklärung eines Bezugsknotens
	\item Die Knotenpotentiale sind die Knotenpotentiale bzgl. des Bezugsknotens
\end{enumerate}

\begin{eqnarray*}
	U_{K_1} & = & -U_2 \\
	U_{K_2} & = & -U_1 \\
	U_{K_3} & = & U_4
\end{eqnarray*}

\paragraph{Knotenpotenzialverfahren}
%graph siehe sonst wer
%modifizierter graph auch
%I_q1 || G_1 || G_4 + G_3 + G_5 || G_2 || I_q2

\subparagraph{Bezugsknoten 1}

Stelle die Knotengleichungen bzgl. der Knotenpotenziale auf.

$K_2$:
\[
	G_1 U_{21} + G_4 U_{21} + I_{q1} = G_3 (U_{31} - U_{21})
\]
\[
	U_{21} ( G_1 + G_4 + G_3 ) - U_{31} G_3 = - I_{q1}
\]

$K_3$:
\[
	G_2 U_{31} + G_5 U_{31} + I_{q2} + G_3 (U_{31} - U_{21}) = 0
\]
\[
	-U_{21} G_3 + U_{31} ( G_2 + G_3 + G_5 ) = -I_{q2}
\]

\[
	\begin{pmatrix}
		G_1 + G_3 + G_4 & - G_3 \\
		- G_3 & G_2 + G_3 + G_4
	\end{pmatrix}
	\begin{pmatrix}
		U_{21} \\
		U_{31}
	\end{pmatrix}
	=
	\begin{pmatrix}
		- I_{q1} \\
		- I_{q2}
	\end{pmatrix}
\]
\[
	G_K^* U_K = I_{qk}^*
\]

Beobachtungen:
\begin{tabular}{rl}
	Matrix $G_K^*$: & Auf der Hauptdiagonalen werden die Summen der am Knoten angeschlossenen Leitwerte eingetragen. \\
	 & Auf der Nebendiagonalen werden die negativen Summen der Leitwerte zwischen den Knoten eingetragen. \\
	Vektor $I_{qk}^*$: & Summe der Quellströme am Knoten. Hinfließend positiv, wegfließend negativ.
\end{tabular}

Beispiel:
%I_q1 || G_1 || G_4 + G_3 + G_5 || G_2 || I_q2

Bezugsknoten 2

\[
	\begin{pmatrix}
		G_1 + G_2 + G_4 + G_5 & -G_2 - G_5 \\
		-G_2 - G_5 & G_2 + G_3 + G_5
	\end{pmatrix}
	\begin{pmatrix}
		U_{12} \\
		U_{32}
	\end{pmatrix}
	=
	\begin{pmatrix}
		I_{q1} + I_{q2} \\
		- I_{q2}
	\end{pmatrix}
\]

Alle Schaltungssimulatoren benutzen das {\it modifizierte} Knotenpotentialverfahren, das auch Spannungsquellen (zuwachs an der Anzahl der Gleichungen) verwenden kann.

\paragraph{Astspannungsanalyse}
%knoten im dreieck, verbindungskanten 4 (ou) und 5 (ou), unten 3 (5-4), außenrum bei 4 1 und bei 5 2
%unten s1 (43) und s2 (53)

\[
	\begin{pmatrix}
		G_1 + G_3 + G_4 & -G_3\\
		-G_3 & G_2 + G_3 + G_5
	\end{pmatrix}
	\begin{pmatrix}
		U_4 \\
		U_5
	\end{pmatrix}
	=
	\begin{pmatrix}
		I_{q1} \\
		I_{q2}
	\end{pmatrix}
\]

Das Knotenpotentialverfahren in Matrixform lautet:
\begin{eqnarray*}
	H \cdot diag(G) H_T U_K & = & - H I_q \\
	U & = & H^T U_K
\end{eqnarray*}

%graph von eben wieder
Knoten 2 Bezugsknoten
\[
	H = \begin{pmatrix}
		-1 & -1 & 0 & 1 & 1 \\
		0 & 1 & 1 & 0 & -1
	\end{pmatrix}
\]
\[
	diag(G) = \begin{pmatrix}
		G_1 & 0 & 0 & 0 & 0 \\
		0 & G_2 & 0 & 0 & 0 \\
		0 & 0 & G_3 & 0 & 0 \\
		0 & 0 & 0 & G_4 & 0 \\
		0 & 0 & 0 & 0 & G_5
	\end{pmatrix}
\]
\[
	H \cdot diag(G) =
	\begin{pmatrix}
		-1 & -1 & 0 & 1 & 1 \\
		0 & 1 & 1 & 0 & -1
	\end{pmatrix}
	\begin{pmatrix}
		G_1 & 0 & 0 & 0 & 0 \\
		0 & G_2 & 0 & 0 & 0 \\
		0 & 0 & G_3 & 0 & 0 \\
		0 & 0 & 0 & G_4 & 0 \\
		0 & 0 & 0 & 0 & G_5
	\end{pmatrix}
	=
	\begin{pmatrix}
		-G_1 & -G_2 & 0 & G_4 & G_5 \\
		0 & G_2 & G_3 & 0 & -G_5
	\end{pmatrix}
\]

%%20.12.2010

%graph siehe block

\[
	U_{q1} = 10V, U_{q2} = 50V, U_{q3} = U_{q4} = U_{q5} = 20V, R_i = 10 \ohm, i = 1, ..., 6
\]
Berechne die Zweigströme

\begin{enumerate}
	\item Rekusives Berechnen, Ersatzquellenmethode: scheiden beide aus.
	\item Überlagerungsverfahren: Aufwand ist die Berechnung von 5 mal 6 Strömen.
	\item Gleichungssystem für die Zweigströme: Inversion einer (6x6)-Matrix.
	\item Maschenstromverfahren: Vollst. Baum - Inversion einer (3x3)-Matrix - Multiplikation mit einer (3x3)-Matrix.
	\item Knotenpotential- und Astspannungsverf.: Ersatzstromquellen mit Innenleitwerten - Inversion einer (3x3)-Matrix - Multiplikation mit einer Matrix - Zweigströme aus Zweigspannungen berechnen.
\end{enumerate}

\subparagraph{Maschenstromverfahren}
Vollst. Baum mit den Zweigen 4, 5 und 6

%bäume siehe block

Sortierfolge
\[
	I_A = \begin{pmatrix}I_4\\ I_5\\ I_6 \end{pmatrix},
	I_S = \begin{pmatrix}I_1\\ I_2\\ I_3 \end{pmatrix}
\]

\[
	R^* \cdot I_S = U_q^*
\]
\[
	\begin{pmatrix}
		R_1 + R_5 + R_6 & -R_5 & -R_6 \\
		-R_5 & R_2 + R_4 + R+5 & -R_4 \\
		-R_6 & -R_4 & R_3 + R_4 + R_6
	\end{pmatrix}
	\begin{pmatrix}
		I_1 \\
		I_2 \\
		I_3
	\end{pmatrix}
	=
	\begin{pmatrix}
		U_{q1} + U_{q5} \\
		U_{q2} + U_{q4} - U_{q5} \\
		U_{q3} + U_{q4}
	\end{pmatrix}
\]

\[
	\begin{pmatrix}
		30 & -10 & -10 \\
		-10 & 30 & -10 \\
		-10 & -10 & 30
	\end{pmatrix} \ohm
	\begin{pmatrix}
		I_1 \\
		I_2 \\
		I_3
	\end{pmatrix}
	=
	\begin{pmatrix}
		30 \\
		50 \\
		0
	\end{pmatrix}
\]

\subparagraph{Gauß-Tableaux}
\[
	\begin{array}{ccc|c}
		30 & -10 & -10 & 30 \\
		-10 & 30 & -10 & 50 \\
		10 & -10 & 30 & 0
	\end{array}
	\rightarrow
	\begin{array}{ccc|c}
		30 & -10 & -10 & 30 \\
		0 & \frac{80}{3} & \frac{-40}{3} & 60 \\
		0 & \frac{-40}{3} & \frac{80}{3} & 10
	\end{array}
	\rightarrow
	\begin{array}{ccc|c}
		30 & -10 & -10 & 30 \\
		0 & \frac{80}{3} & \frac{-40}{3} & 60 \\
		0 & 0 & 20 & 40
	\end{array}
\]
\[
	\rightarrow I_3 = 2A, I_2 = 3,25A, I_1 = 2,75A
\]

Astströme - $I_A = M_A^T I_S$ oder Knotengleichungen

Knotengleichungen
\begin{eqnarray*}
	K_1: I_6 & = & I_1 - I_3 \\
	K_2: I_5 & = & I_2 - I_1 \\
	K_4: I_4 & = & I_2 - I_3
\end{eqnarray*}

\[
	\begin{pmatrix}
		0 & 1 & -1 \\
		-1 & 1 & 0 \\
		1 & 0 & -1
	\end{pmatrix}
	\begin{pmatrix}
		I_1 \\
		I_2 \\
		I_3
	\end{pmatrix}
	=
	\begin{pmatrix}
		I_4 \\
		I_5 \\
		I_6
	\end{pmatrix}
	=
	\begin{pmatrix}
		1,25 A \\
		0,5 A \\
		0,75 A
	\end{pmatrix}
\]

Astpartition der Maschenmatrix

\underline{Sortierfolge: 4, 5, 6, 1, 2, 3}

\[
	M = (M_A | M_S) = (M_A | E) =
	\begin{pmatrix}
		0 & -1 & 1 & | & 1 & 0 & 0 \\
		1 & 1 & 0 & | & 0 & 1 & 0 \\
		-1 & 0 & -1 & | & 0 & 0 & 1
	\end{pmatrix}
\]
\[
	M_A = \begin{pmatrix}
		0 & -1 & 1 \\
		1 & 1 & 0 \\
		-1 & 0 & -1
	\end{pmatrix},
	M_A^T = \begin{pmatrix}
		0 & 1 & -1 \\
		-1 & 1 & 0 \\
		1 & 0 & -1
	\end{pmatrix}
\]

%umgebautes netz siehe block

\begin{tabular}{c|c|c|c|c}
	Zweig j & $U_{qj}$ & $R_j$ & $I_{qj}$ & $G_j$ \\
	1 & 10 V & 10 $\ohm$ & 1 A & 150 mS \\
	2 & 50 V & 10 $\ohm$ & 5 A & 150 mS \\
	3 & 20 V & 10 $\ohm$ & 2 A & 150 mS \\
	4 & 20 V & 10 $\ohm$ & 2 A & 150 mS \\
	5 & 20 V & 10 $\ohm$ & {\color{yellow} -2 A} & 150 mS \\
	6 & 0 & 10 $\ohm$ & 0 & 150 mS \\
\end{tabular}

Bezugsknoten: 3

\[
	U_K = \begin{pmatrix}
		U_{K1} \\
		U_{K2} \\
		U_{K3}
	\end{pmatrix} = \begin{pmatrix}
		U_{13} \\
		U_{23} \\
		U_{43}
	\end{pmatrix} = \begin{pmatrix}
		- U_6 \\
		- U_5 \\
		U_4
	\end{pmatrix}
\]

Knotenpotentialverfahren: $G_K^* U_K = I_{Kq}^*$

\[
	\begin{pmatrix}
		G_1 + G_3 + G_6 & -G_1 & -G_3 \\
		-G_1 & G_1 + G_2 + G_5 & -G_2 \\
		-G_3 & -G_2 & G_2 + G_3 + G_4
	\end{pmatrix}
	\begin{pmatrix}
		U_{K1} \\
		U_{K2} \\
		U_{K3}
	\end{pmatrix}
	=
	\begin{pmatrix}
		-I_{q1} + I_{q3} \\
		I_{q1} - I_{q2} + I_{q5} \\
		I_{q2} - I_{q4} - I_{q3}
	\end{pmatrix}
\]
\[
	S \begin{pmatrix}
		0,3 & -0.1 & -0,1 \\
		-0,1 & 0.3 & -0,1 \\
		-0,1 & -0.1 & 0,3
	\end{pmatrix}
	\begin{pmatrix}
		U_{K1} \\
		U_{K2} \\
		U_{K3}
	\end{pmatrix}
	=
	\begin{pmatrix}
		1 A \\
		-6 A \\
		1 A
	\end{pmatrix}
\]
\[
	\begin{pmatrix}
		U_{13} \\
		U_{23} \\
		U_{43}
	\end{pmatrix}
	=
	\begin{pmatrix}
		-7,5 V \\
		-25 V
		-7,5 V
	\end{pmatrix}
\]

Fehlende Spannungen - Ablesen oder $U = H^T U_K$

%graph siehe block

\begin{eqnarray*}
	U_1 & = U_{13} - U_{23} = & 17,5V \\
	U_2 & = U_{23} - U_{43} = & -17,5V \\
	U_3 & = - U_{13} + U_{43} - & 0 \\
	U_4 & = U_{43} = & -7,5 V \\
	U_5 & = - U_{23} = & 25V \\
	U_6 & = - U_{13} = & 7,5V
\end{eqnarray*}

\[
	I_j = U_j G_j + I_{qj} = \left( \begin{pmatrix}
		17,5 \\
		-17,5 \\
		0 \\
		-7,5 \\
		25 \\
		7,5
	\end{pmatrix} 0,1 + \begin{pmatrix}
		1 \\
		5 \\
		2 \\
		2 \\
		-2 \\
		0
	\end{pmatrix} \right) A
\]
\[
	I = \begin{pmatrix}
		2,75 \\
		3,25 \\
		2,0 \\
		1,25 \\
		0,5 \\
		0,75
	\end{pmatrix}
\]

Methode 2: H-Matrix

\[
	H = \begin{pmatrix}
		0 & 0 & -1 & 1 & 0 & -1 \\
		0 & -1 & 0 & -1 & 1 & 0 \\
		1 & 0 & 0 & 0 & -1 & 1
	\end{pmatrix}
\]
\[
	U = \begin{pmatrix}
		0 & 0 & 1 \\
		0 & -1 & 0 \\
		-1 & 0 & 0 \\
		1 & -1 & 0 \\
		0 & 1 & -1 \\
		-1 & 0 & 1
	\end{pmatrix}
	\begin{pmatrix}
		U_{K1} \\
		U_{K2} \\
		U_{K3}
	\end{pmatrix}
	=
	\begin{pmatrix}
		U_{43} \\
		- U_{23} \\
		- U_{13} \\
		U_{13} - U_{23} \\
		U_{23} - U_{43} \\
		- U_{13} + U_{43}
	\end{pmatrix}
	=
	\begin{pmatrix}
		U_4 \\
		U_5 \\
		U_6 \\
		U_1 \\
		U_2 \\
		U_3
	\end{pmatrix}
\]

Astspannungsverfahren

%graph siehe block (nr 5)

\[
	S = ( S_A | S_S ) = \begin{pmatrix}
		1 & 0 & 0 & | & 0 & -1 & 1 \\
		0 & 1 & 0 & | & 1 & -1 & 0 \\
		0 & 0 & 1 & | & -1 & 0 & 1
	\end{pmatrix}
\]

\[
	G^* U_A = I_q^*, G^* = S \cdot diag(G) S^T
\]

$G^*: \begin{array}{l}
	Hauptdiagonalen: Summe der am Superknoten angeschlossenen Leitwerte \\
	Nebendiagonalen: Summe der Leitwerte zwischen den Superknoten (liegen auf Sehnen), positiv bei gleicher Schnittorientierung, negativ bei ungleicher Schnittorientierung
\end{array}$

$I_q^*$: Summe der Quellströme an den Superknoten, negativ wenn Stromrichtung und Schnittorientierung gleich, positiv sonst

\[
	G^* = \begin{pmatrix}
		G_2 + G_3 + G_4 & G_2 & G_3 \\
		G_2 & G_1 + G_2 + G_5 & -G_1 \\
		G_3 & -G_1 & G_1 + G_3 + G_6
	\end{pmatrix}
\]

\[
	I_q^* = \begin{pmatrix}
		I_{q2} - I_{q3} - I_{q4} \\
		- I_{q1} - I_{q5} + I_{q2} \\
		I_{q1} - I_{q3}
	\end{pmatrix}
\]

\[
	U_A = \left( G^* \right)^{-1} I_q^*, U_S = S_S^T U_A
\]

\[
	S \cdot diag(G) \cdot S^T = \begin{pmatrix}
		1 & 0 & 0 & 0 & -1 & 1 \\
		0 & 1 & 0 & 1 & -1 & 0 \\
		0 & 0 & 1 & -1 & 0 & 1
	\end{pmatrix}
	\begin{pmatrix}
		G_4 & 0 & 0 & 0 & 0 & 0 \\
		0 & G_5 & 0 & 0 & 0 & 0 \\
		0 & 0 & G_6 & 0 & 0 & 0 \\
		0 & 0 & 0 & G_1 & 0 & 0 \\
		0 & 0 & 0 & 0 & G_2 & 0 \\
		0 & 0 & 0 & 0 & 0 & G_3
	\end{pmatrix}
	\begin{pmatrix}
		1 & 0 & 0 \\
		0 & 1 & 0 \\
		0 & 0 & 1 \\
		0 & 1 & -1 \\
		-1 & -1 & 0 \\
		1 & 0 & 1
	\end{pmatrix}
\]

%% 07.01.2011

\section{Netzwerke mit nichtlinearen Bauelementen}

Drei Berechnungsverfahren
\begin{enumerate}
	\item Der Kennlinienverlauf wird durch eine geeignete mathematische Funktion genähert und das Netzwerk mit Hilfe von Knoten- und Maschengleichungen analysiert.
		Die auftretenden Gleichungen sind nichtlinear.
	\item Die Kennlinie des nichtlinearen Bauelements wird in der Umgebung eines {\it vermuteten} Arbeitspunkts linearisiert. Hier ist es oft nötig, nach der Analyse die Abweichung vom Verhalten des Bauelements nachzuprüfen und die Vermutungen iterativ zu verbessern.
	\item Im NW mit nur einem nichtlinearen Bauelement und beliebig vielen linearen Bauelementen ist eine graphische AP-Bestimmung mit einem Ersatzzweipolquellenverfahren möglich.
	% TODO: Graphen siehe Daniel
		Wir bringen beide Kennlinien zum Schnitt, d.h. wir lösen $U_L(I) = U_{NL}(I)$
		{\color{yellow} AP: $(I^*, U^*)$, für den gilt: $U_{NL}(I^*) = U_L(I^*) = U_0 - R_i I^*$
\end{enumerate}

Bsp. Linearisierungsverfahren
% diode (A, K)
Lineare Ersatzschaltung
% R_{iE}
% U_{oE}

%tabelle plus graph
U_D; I_D
0,6; 0
0,7; 1
0,8; 5
0,9; 17
0,95; 25

Für die Linearisierung wählen wir einen Punkt $P_L: (U_L, I_L) = (0,82V, 7A)$

\underline{Ablesen:} $R_{iE} = 7,4m\ohm, U_{oE} = 0,78V$

Die Ersatzschaltung repräsentiert die Diode mit brauchbarer Übereinstimmung im Strombereich zwischen $7A$ und $30A$.

Wir überprüfen die Linearisierung für das NW
(i) $R_i = 10m\ohm$ (ii) $R_i = 100m\ohm$

Linearisierter Fall
$I_{DE} = \frac{U_0 - U_{oE}}{R_i + R_{iE}}$
$U_{DE} = U_{oE} + I_{DE} R_{iE}$

\begin{tabular}{c|c|c|c|c}
	 & \multicol{2}{Reale Diode} & \multicol{2}{Ersatzschaltung} \\
	 & $I_D/A$ & $U_D/V$ & $I_{DE}/A$ & $U_{DE}/V$ \\
	R_i = 10m\ohm & \ungefähr 12 & \ca 0,88 & 12,64 & 0,87 \\
	R_i = 100m\ohm & \ca 3 & \ca 0,78 & 2,05 & 0,80
\end{tabular}

Anwendung einer Zener-Diode zur Spannungsstabilisierung
%schaltung siehe Daniel

Ersatzschaltung
%schaltung siehe Daniel

$U_0 = 30V, R_L = 2200\ohm, U_{oE} = 5,6V, R_{iE} = 10\ohm$

\begin{enumerate}
	\item Bestimmen sie einen Vorwiderstand $R_v$ so, dass sich ein Z-Diodenstrom von 3,5mA einstellt. (Das ist der AP)
	\item Wie ändert sich die Lastspannung $U_L$, wenn sich die Eingangsspannung $U_0$ um $10 \percent$ ändert?
	\item Wie ändert sich die Lastspannung $U_L$, wenn sich der Lastwiderstand $R_L$ um $10 \percent$ ändert?
\end{enumerate}

Ersatzschaltung

%schaltung siehe daniel
$I = I_Z + I_L$
Da $I_Z R_{iE} << U_{oE}$ ist, setzen wir $U_L \ca $U_{oE}$.

$I_L = \frac{U_{oE}}{R_L} = \frac{5,6V}{2,2 k\Ohm} = 2,55 mA$
$I = I_L + I_Z = 2,55 mA + 3,5 mA = 6,05 mA$
$R_v = \frac{U_0 - U_{oE}}{I} = 4,03 k\Ohm$

Überprüfung der Vernachlässigung
$U_Z = U_{oE} + R_{iE} I_Z = U_{oE} + 35mV = 5,6{\it 35} V$

(2) Gesucht wird der funktionale Zusammenhang $U_Z = U_L = f(U_0)$

KGl: $I = I_Z + I_L$ \\
MGl_1: $U_0 = IR_V + I_Z R_{iE} + U_{oE}$ \\
MGl_2: $U_Z = I_Z R_{iE} + U_{oE} = I_L R_L = U_L

$\Rightarrow U_0 = U_L \left( \frac{R_V}{R_L} + \frac{R_V + R_{iE}}{R_{iE}} \right) + U_{oE} \left( 1 - \frac{R_V + R_{iE}}{R_{iE}} \right)$

$\delta U_L = \delta U_0 \frac{1}{\frac{R_V}{R_L} + \frac{R_V + R_{iE}}{R_{iE}}}$ $\delta U_0 = 0,1 \cdot U_0 = 3V$ $\frac{\delta U_0}{U_0} = 0,1$
$\frac{\delta U_L}{U_L} = \frac{\delta U_0}{U_L} \frac{1}{\frac{R_V}{R_L} + \frac{R_V + R_{iE}}{R_{iE}}} = 0,0013 = 0,13 \percent$

%% 10.01.2011 fehlt

%% 14.01.2011

\subsection{Leistung}
\begin{enumerate}
	\item $I$ und $U$ sind konstant oder Effektivwerte, wenn Strom und Spannung proportional zueinander sind, z.B. Strom und Spannung sind in Phase bei sinusförmigen Größen.
		\[ P = \frac{W}{t} = U \cdot I \]
	\item Augenblicksleistung
		\[ p(t) = u(t) \cdot i(t) \]
	\item Wirkleistung
		\[ P = \={\p} = \frac1T \int_{t_0}^{t_0 + T} p(t) dt = \frac1T \int_{t_0}^{t_0 + T} u(t) \cdot i(t) dt \]
	\item Scheinleistung, Strom und Spannung sind nicht mehr proportional
		\[ S = U \cdot I = \sqrt{\frac1T \int_{t_0}^{t_0 + T} u^2 (t) dt} \cdot \sqrt{\frac1T \int_{t_0}^{t_0 + T} i^2 (t) dt} \]
	\item Komplexe Leistung $\underline P$
		\[ P = Re\{\underline P\} \text{: Wirkleistung, } [P] = W \]
		\[ Q = Im\{\underline P\} \text{: Blindleistung, } [Q] = VA_r \]
		\[ |\underline P| = \sqrt{P^2 + Q^2} = S \text{: Scheinleistung, } [S] = VA \]
\end{enumerate}

\subsection{Sinusgrößen}
%diagramm siehe daniel

\begin{eqnarray*}
	i(t) & = & \^{\i} \cdot sin(\omega t + \phi_i) \\
	u(t) & = & u^ \cdot sin(\omega t + \phi_u) \\
	\phi = \phi_u - \phi_i \text{: Phasenwinkel}
\end{eqnarray*}
$\phi_i, \phi_u$: Winkel zum \underline{nächst}gelegenen Nulldurchgang mit anwachsender Funktion.

$\phi_i > 0, \phi_u < 0, \phi < 0$

\subsection{Sprachgebrauch}
$\phi < 0$: Der Strom eilt der Spannung voraus oder die Spannung eilt dem Strom nach \\
$\phi > 0$: Der Strom eilt der Spannung nach oder die Spannung eilt dem Strom vor

\subsection{Mittelwerte von Sinusgrößen}
Arith. Mittelwert:
	\[ \={\i} = \frac1T \int_0^1 \^{\i} \cdot sin(\omega t) dt = \frac{i^}T \left( \left. \frac{- cos(\omega t)}\omega \right|_0^1 \right) = \frac{i^}T \left( cos(0) - \underbrace{cos(\omega T)}_{cos(2\pi)}) = 0 \]

Gleichrichtwert:
	\[ | \={\i} | = \frac1T \int_0^1 |i(t)| dt = \frac1{\fracT2} \int_0^{\fracT2} \^{\i} sin(\omega t) dt = \ldots = \frac{2i^}\pi = 0,6366 \^{\i} \]

Effektivwert:
	\[ I = \sqrt{ \frac1T \int_0^T \left( \^{\i} \cdot sin(\omega t) \right)^2 dt } = \sqrt{ \frac1T \underbrace{\int_0^T \frac{{i^}^2}2 dt}_{\frac{i^}2 \cdot T} - \underbrace{\frac1T \int_0^T \frac12 {i^}^2 \cdot cos(2\omega t) dt}_{=0 \text{arith. Mittelwert einer Wechselgröße}} } = \frac{i^}{\sqrt{2}} ~= 0,7071 i^\]
	Spitzenwertfaktor: $\tchi = \frac{i^}I = \sqrt{2}$ \\
	Formfaktor: $F = \fracI{|\={\i}|} = \frac{\pi}{2 \sqrt{2}} = 1,111$

Bsp. %siehe Daniel
Stromwinkel $0 \leq \alpha \lqq 150\grad \frac{\pi}{180\grad}$
Effektivwert der Spannung: $U = 220V$
Zwischen welchen Grenzen läßt sich der Effektivwert $I$ einstellen, wenn $\alpa$ zwischen $0$ und $\pi \frac{150}{180}$ gewählt wird?

%TODO (zu lange rechnung)

$I(\alpha = 0) = 1,56 A$, $I(\alpha = \pi \frac{150}{180}) = 0,26A$

Parameter sinusförmiger Größen, Zeigerdiagramme
\[ i(t) = \^{\i} \cdot sin(\omega t + \phi i) \]
Parameter: Kreisfrequenz $\omega$, $\omega = 2 \pi f$, $f:$ Frequenz
Spitzenwert $\^{\i}$
	$\rightarrow$ Effektivwert $I = \frac{\^{\i}}{\sqrt{2}}$
	$\rightarrow$ Gleichrichtwert $|\={\i}| = 0,6366 \^{\i}$
	$\rightarrow$ ...faktor $\tchi = \sqrt{2}$
	$\rightarrow$ Formfaktor $F = 1,111$
Phasenwinkel $\phi_i$

Die symbolische Darstellung erfolgt mit dem Zeigerdiagramm.

Wir analysieren in GLET1 {\it lineare} NW und gehen davon aus, dass {\it alle} Quellen {\it sinusförmige} Spannungen oder Ströme mit {\it derselben Frequenz} erzeugen. Dann sind alle Zweigspannungen und Zweigströme sinusförmig mit ebendieser Frequenz.
\[ \sum_{\nu} \^{\u}_{\nu} sin( \omega_0 t + \phi_{\nu} ) = \^{\u} \cdot sin( \omega_0 t + \phi ) \]
komplex:
\[ \sum_{\nu} a_{\nu} e^{j(\omega_0 t + \phi_{\nu})} = \sum_{\nu} a_{\nu} e^{... \] %TODO

%% 17.01.2011
\subsection{Addition und Subtraktion von Sinusgrößen}

Gegeben sind zwei Spannungen mit gleicher Frequenz und wir berechnen die Summenspannung.

\begin{eqnarray}
	u_g(t) & = & u_1(t) + u_2(t) \\
	\^{\u}_g \cdot sin(\omega t + \phi_g) & = & \^{\u}_1 \cdot sin(\omega t + \phi_1) + \^{\u}_2 \cdot sin(\omega t + \phi_2)
\end{eqnarray}
nach ermüdender Rechnung erhält man
\[ \^{\u}_g = \sqrt{\^{\u}_1^2 + \^{\u}_2^2 + 2 \^{\u}_1 \^{\u}_2 cos(\phi_2 - \phi_1)}, \phi_g = arctan\left( \frac{\^{\u}_1 \cdot sin(\phi_1) + \^{\u}_2 \cdot sin(\phi_2)}{\^{\u}_1 \cdot cos(\phi_1) + \^{\u}_2 \cdot cos(\phi_2) } \right) \]

Spitzenwertzeiger $\underline{\^{\u}}$ \\
Effektivwertzeiger $\underline{U} = \frac{\underline{\^{\u}}}{\sqrt{2}}$

Bsp. 
...
a) und b) schaltung mit 2 spannungsquellen u_1 und u_2, bei a gleiche richtung, bei b entgegengesetzt

Zur Vereinfachung der Rechnung setzen wir einen Winkel zu Null, z.B. $\phi_1 = 0$ \\
a) \[U_g = \sqrt{U_1^2 + U_2^2 + 2 U_1 U_2 cos(\phi_2 - \phi_1)} = \sqrt{50^2 + 30^2 + 2 \cdot 50 \cdot 30 \cdot cos(60\grad)} = 70V \]
\[ \phi_{1g} = arctan\left( \frac{50 \cdot sin(\phi_1) + 30 \cdot sin(\phi_2)}{50 \cdot cos(\phi_1) + 30 \cdot cos(\phi_2)} \right) = arctan\left( \frac{30 \cdot sin(60\grad)}{50 + 30 \cdot cos(60\grad)} \right) = 21,79\grad \]

b) \[U_g = \sqrt{U_1^2 + U_2^2 + 2 U_1 U_2 cos(\phi_2 - \phi_1)} = \sqrt{50^2 + 30^2 + 2 \cdot 50 \cdot 30 \cdot cos(-120\grad)} = 43,59V \]
\[ \phi_{1g} = arctan\left( \frac{50 \cdot sin(\phi_1) + 30 \cdot sin(\phi_2)}{50 \cdot cos(\phi_1) + 30 \cdot cos(\phi_2)} \right) = arctan\left( \frac{30 \cdot sin(-120\grad)}{50 + 30 \cdot cos(-120\grad)} \right) = -36,59\grad \]

\subsection{Differentiation und Integration von Sinusgrößen}
\paragraph{Differentiation nach der Zeit}
\[ f(t) = \frac{d}{dt} ( \^{\i} \cdot sin(\omega t + \phi_i) ) = \omega \cdot \^{\i} \cdot cos(\omega t + \phi_i) = \omega \cdot \^{\i} \cdot sin(\omega t + \phi_i + \frac{\pi}2 ) \]
\begin{itemize}
	\item Wir erhalten eine Sinusschwingung mit derselben Frequenz, die der ursprünglich um $\frac{\pi}2$ oder $90\grad$ voreilt.
	\item Die Amplitude entspricht dem $\omega$-fachen der Ursprungsamplitude.
\end{itemize}

\paragraph{Integration über der Zeit}
\[ g(t) = \int \^{\i} \cdot sin(\omega t + \phi_i) dt = \frac{-1}{\omega} \^{\i} cos(\omega t + \phi_i) = \frac1{\omega} \^{\i} \cdot sin(\omega t + \phi_i - \frac{\pi}2) \]
\begin{itemize}
	\item Wir erhalten eine Sinusschwingung mit der selben Frequenz, die der ursprünglichen um $\frac{\pi}2$ oder $90\grad$ nacheilt.
	\item Die Amplitude entspricht dem $\left( \frac1{\omega} \right)$-fachen der Ursprungsamplitude.
\end{itemize}

Induktivität
%kasten mit spule (Kennzeichen: L)
$L$: Induktivität
$[L] = \frac{V_s}A = H$

%schaltung mit einer Induktivität L, der Spannung \underline{U} und dem Strom i

Allg. $u = L \cdot \frac{di}{dt}$ $i = \frac1L \int u dt$

Sinusförmige Größen
Sei $i(t) = \^{\i} \cdot sin(\omega t + \phi_i)$, dann ist $u(t) = L \cdot \frac{d}{dt} i(t) = L \cdot \omega \cdot \^{\i} \cdot sin(\omega t + \phi_i + \frac{\pi}2) = \^{\u} \cdot sin(\omega t + \phi_u)$, $\^{\u} = L \omega \^{\i}$, $\phi_u = \phi_i + \frac{\pi}2$

Kondensator, Kapazität
%konsensator schaltzeichen

%schaltung mit strom i_t, einer Spannung u_t
$i = c \cdot \frac{du}{dt}$
$u = \frac1c \cdot \int i dt$

Sinusförmige Größen
Sei $u(t) = \^{\u} \cdot sin(\omega t + \phi_u)$, dann ist $i(t) = \^{\u} \cdot c \cdot \omega \cdot sin(\omega t + \phi_u + \frac{\pi}2) = \^{\i} \cdot sin(\omega t + \phi_i)$ mit $\^{\i} = \^{\u} \cdot c \cdot \omega$ und $\phi_i = \phi_u + \frac{\pi}2$

Rechenbeispiel
%u(t) + i(t) + ((i_R(t) + R) || (i_L(t) + L))

kGl: $i(t) = i_L(t) + i_R(t)$

\begin{enumerate}
	\item Augenblicksleistung in der Parallelschaltung
	\item Wrikleistung im Widerstand
	\item Scheinleistung in der Induktivität
\end{enumerate}

$u(t) = \^{\u} \cdot(\omega t)$, $t \geq 0$, $i_L(t=0) = 0$

\begin{enumerate}
	\item \[ i_R(t) = \frac{u(t)}{R} = \frac{\^{\u} \cdot sin(\omega t)}{R} \]
				\[ i_L(t) = \frac1L \int u(t) dt = \frac{\^{\u}}{L} \int sin(\omega t) dt = \frac{-\^{\u}}{\omega L} cos(\omega t) \]
				\[ i(t) = \frac{\^{\u}}R sin(\omega t) - \frac{\^{\u}}{\omega L} cos(\omega t) \]
				\[ p(t) = u(t) i(t) = \^{\u} \cdot sin(\omega t) \left( \frac{\^{\u}}R sin(\omega t) - \frac{\^{\u}}{\omega L} cos(\omega t) \right) = \frac{\^{\u}^2}R sin^2(\omega t) - \frac{\^{\u}^2}{2 \omega L} sin(2 \omega t) = \frac{\^{\u}^2}R \cdot \frac12 (Gleichanteil) - \frac{\^{\u}^2}{2R} cos(2 \omega t) (Wechselanteil)  - \frac{\^{\u}^2}{2\omega L} sin(2 \omega t) (Wechselanteil) \]
	\item \[ P = \frac1{\pi} \int_{t_0}^{t_0 + T} p(t) dt = \frac{\^{\u}^2}{2R} = \frac{U^2}R \]
	\item Scheinleistung in der Induktivität
		\[ S = U \cdot I = \sqrt{ \frac1T \int_{t_0}^{t_0 + T} \^{\u}^2 \cdot sin^2(\omega t) dt} \cdot \sqrt{ \frac1T \int_{t_0}^{t_0 + T} \frac{\^{\u}^2}{(\omega L)^2} cos^2(\omega t) dt} \cdot \frac{\^{\u}}{\sqrt{2}} \cdot \frac{\^{\u}}{\sqrt{2} \cdot \omega \cdot L} = \frac{U^2}{\omega L} \]
		$\omega L = |Z_L|$ Betrag der Impedanz der Induktivität
\end{enumerate}

\paragraph{Komplexe Zeiger}
Euler: $\underline{z} = e^{jx} = cos(x) + j \cdot sin(x)$
$Re\{\underline{z}\} = \frac12 (\underline{z} + \underline{z}^*)$
$cos(x) = \frac12 (e^{jx} + e^{-jx})$
$Im\{\underline{z}\} = \frac1{2j} (e^{jx} - e^{-jx})$
$sin(x) = \frac1{2j} (e^{jx} - e^{-jx})$

De Moivre:
(cos(\phi) + j sin(\phi))^n = cos(n\phi) + j \cdot sin(n \phi)$

komplexe Schwingung $\underline{u} = \^{\u} \cdot e^{j(\omega t + \phi_u)}$
Zeitfunktion $u(t) = \^{\u} \cdot sin(\omega t + \phi_u) = Im\{\underline{u}\}

Differentiation nach der Zeit $\rightarrow$ Multiplikation mit $j \omega = \omega \cdot e^{j \frac{\pi}2}$
Integration über der Zeit $\rightarrow$ Multiplikation mit $\frac1{j \omega} = -j \frac1{\omega} = \frac1{\omega} e^{-j \frac{\pi}2}$

Bsp.: Gesucht sind die komplexen Zahlen $\underline{z_1} = x_1 + jy_1$ und $\underline{z_2} = x_2 + jy_2$, die zueinander reziprok sind und die Realteile $x_1 = 0,3$ sowie $x_2 = 1,2$ haben.
$\underline{z_1} = 0,3 + jy_1 = \frac1{\underline{z_2}} = \frac1{1,2 + jy_2}$

konjugiertkomplex erweitern: $\frac1{\underline{z}} = \frac{\underline{z^*}}{\underline{z}\underline{z^*}} = \frac{\underline{z^*}}{|\underline{z}|^2}
$\frac1{1,2 + jy_2} = \frac{1,2 - jy_2}{(1,2)^2 + y_2^2}$
Realteile $0,3 = \frac{1,2}{(1,2)^2 + y_2^2} \rightarrow y_2 = \pm 1,6$
Imaginärteile $y_2 = \frac{-y_2}{(1,2)^2 + y_2^2} \rightarrow y_2 = \mp 0,4$

$\underline{z_1} = 0,3 - j0,4$
$\underline{z_2} = 1,2 + j1,6$
oder
$\underline{z_1} = 0,3 + j0,4$
$\underline{z_2} = 1,2 - j1,6$

Bsp.: Für $f = 50 Hz$ und $\phi_u = 60 \grad$ sollen bei $U = 230V$ der Spitzenwert $\^{\u}$ berechnet und die komplexe Darstellung angegeben werden. Zudem soll $u(t)$ an der Stelle $t = 12ms$ berechnet werden.
reelle Darstellung: $u(t) = \^{\u} \cdot sin(\omega t + \phi_u)$
komplexe Darstellung: $\underline{u} = \^{\u} \cdot e^{j(\omega t + \phi_u)}
Zusammenhang: $u(t) = Im\{\underline{u}\}$
\[ \^{\u} = \sqrt{2} \cdot U = 325,3V \]
\[ \underline{u} = 325,3V \cdot e^{j(2\pi \cdot 50 \cdot \frac1s \cdot t - \frac{\pi}3)} = 325,3V \cdot e^{j(314 \frac1s - \frac{\pi}3)} \]
\[ t(t = 12ms) = Im\{ 325,3V \cdot e^{j(314 \frac{0,012s}{s} \cdot \frac{\pi}3)}\} = 325,3V \cdot sin(314 \cdot 0,012 - \frac{\pi}3) = 132,3V \]

## 21.01.2011
\subsection{Komplexe Größen zur Analyse von Sinusstrom-NW}

\paragraph{Komplexer Drehzeiger}
\[ \underline{u} = \^{\u} e^{j(\omega t + \phi_u)} = \^{\u} e^{j\phi_u} e^{j\omega t} \]
$\underline{u}$: ruhender Zeiger zum Zeitpunkt $t=0$ \\
Komplexe Amplitude: $\underline{\^{\u}} = \^{\u} \cdot e^{j\phi_u}$, $\underline{u} = \underline{\^{\u}} e^{j\omega t}$ \\
Die komplexe Amplitude entspricht dem rudenen Drehzeiger zum Zeitpunkt $t=0$ \\
Im zeitinvarianten NW ändern sich die Verhältniss der Zweiggrößen zueinander nicht mit der Zeit, somit können wir sie zu $t=0$ setzen.

Effektivwertzeiger $U = \frac{\^{\u}}{\sqrt{2}}$, $\underline{u} = U \cdot e^{j\phi_u}$, $\underline{I} = I \cdot e^{j\phi_i}$

$\phi = \phi_u - \phi_i$: Phasendifferenz \\
$\phi_u, \phi_i$: Nullphasenwinkel

\paragraph{Komplexe Widerstände (Impedanzen) und komplexe Leitwerte (Admittanzen)}
\[ \underline{z} = \frac{\underline{U}}{\underline{I}} = \frac{U e^{j\phi_u}}{I e^{j\phi_i}} = \frac{U}{I} e^{j(\phi_u - \phi_i)} = \frac{U}{I} e^{j\phi} = z e^{j\phi} \hspace z = \frac{U}{I} \]
\[ \underline{z} = R + jX, |\underline{z}| = z = \sqrt{R^2 + X^2}, \phi = arctan\left( \frac{X}{U} \right) \]

$|\underline{z}| = z = \sqrt{R^2 + X^2}$: Scheinwiderstand \\
$R = \underline{z} cos(\phi) = Re\{\underline{z}\}$: Wirkwiderstand \\
$X = \underline{z} sin(\phi) = Im\{\underline{z}\}$: Blindwiderstand, Reaktanz

Komplexer Leitwert: $\underline{y} = \frac{\underline{I}}{\underline{U}} = \frac{I \cdot e^{j\phi_i}}{U \cdot e^{j\phi_u}} = \frac{I}{U} e^{j(\phi_i - \phi_u)} = \frac{I}{U} e^{-j\phi} = \frac{I}{U} e^{j\phi_y}$
Kartesische Koordinaten: $\underline{y} = G + jB$, $\phi_y = arctan\left( \frac{R}{G} \right)$ \\
$\undeline{y}$: komplexer Leitwert, $Y = |\underline{y}| = \sqrt{G^2 + R^2}$: Scheinleitwert \\
$G = Re\{\underline{y}\} = \underline{y} cos(\phi_y)$: Wirkleitwert, Konduktanz \\
$B = Im\{\underline{y}\} = \underline{y} sin(\phi_y)$: Blindleitwert, Suszeptanz

Zusammenhang zwischen $\underline{z}$ und $\undelrine{y}$: $\underline{y} = \frac1{\underline{y}} = \frac1z e^{-j\phi}$, $\underline{y} = \frac1z e^{j\phi_y}$ mit $\phi_y$ = -\phi$

Rechnen mit kartesischer Darstellung
\[ \underline{y} = \frac1{\underline{z}} = G + jB = \frac1{R + jX} = \frac{R - jX}{R^2 + X^2} = \frac{R}{R^2 + X^2} - j\frac{X}{R^2 - X^2} = \frac{R}{z^2} - j \frac{X}{z^2}\]
\[ \rightarrow G = \frac{R}{R^2 + X^2} \rightarrow B = \frac{-X}{R^2 + X^2} \]
$G = \frac1R$, gültig für $X = 0$

\[ \underline{z} = \frac1{\underline{y}} = R + jX = \frac1{G + jB} = \frac{G}{G^2 + B^2} - j\frac{B}{G^2 - B^2} = \frac{G}{y^2} - j \frac{B}{y^2}\]
\[ \rightarrow R = \frac{G}{G^2 + B^2} = \frac{G}{y^2} \rightarrow X = \frac{-B}{G^2 + B^2} = \frac{-B}{y^2} \]

Komplexe Leistung
Der ohmsche Verbraucher $R$ nimmt die Wirkleistung $P = RI^2 = GU^2$ auf. Eine komplexe Leistung $\underline{S} = \underline{Z}I^2 = RI^2 + jXI^2 = P + jQ$

$\underline{s} = \underline{y} {\it \cdot} U^2 = GU^2 {\it -} jBU^2$ \hspace $\winkel \underline{s} = \winkel \underline{z} = - \winkel \underline{y}$

$\underline{S}$: komplexe Leistung, $|\underline{S}| = S = UI$: Scheinleistung, $Q = XI^2 = {\it -} BU^2$: Blindleistung

$\underline{S} = \underline{U} \underline{I}^*$, beim ohmschen Verbraucher sind Strom und Spannung in Phase.

$\pni_u = \phi_i = \~\phi$, $\underline{S} = \underline{U} \underline{I}^* e^{j\~\phi} e^{-j\~\phi} = UI$

#TODO: hier fehlt was

Leistungsfaktor: $cos(\phi) = \frac{P}{\sqrt{P^2 + Q^2}} = \frac{R}{\sqrt{R^2 + X^2}}$

\begin{tabular}{cccccc}
	 & Zeitfunktionen & Eff.Zeiger & komplexe Größen & Impedanzen & komplexe Leistung \\
	R(wiederstand) & $u = Ri$ $i = \frac1R u$ & $\underline{U} = R\underline{I}$ $\underline{I} = \frac1R \underline{U}$ & (in phase) & 
#am arsch
\end{tabular}

Bsp. An einer Spule mit vernachlässigbarem Wirkwiederstand (Kupfer-, Eigenverlust) liegt eine Wechselspannung mit $f = 40Hz$ und $U = 125V$ an. Der Effektivstrom beträgt $I = 10A$. Wie groß ist die Spuleninduktivität $L$?

$\underline{z} = j\omega L = j \frac{U}{I} ($\underline{z} = \frac{\underline{U}}{\underline{I}} = \frac{U}{I} e^{j(\phi_u - \phi_i)} = \frac{U}{I} j )$

$L = \frac{U}{\omega I} = \frac{125V}{2 \pi \cdot 40 \cdot \frac1s \cdot 10A} = 49,74mH$

#noch ein bleistift
\end{document}
